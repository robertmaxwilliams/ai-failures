\documentclass[11pt]{article}
\usepackage{cite}
\usepackage[useregional]{datetime2}
\usepackage{babel}
\usepackage{dirtytalk}



\begin{document}

\title{Data Driven AI Failure Avoidance}
\author{Max Williams \\ University of Louisville \and Roman Yampolskiy \\ University of Louisville}
\date{\today}
\maketitle

\abstract{
As we assign tasks of greater importance to AI in greater quantity, the rate and severity of both
benign and catastrophic AI failures increases. In this paper, we review data collected on AI
failures and discover trends in their occurance. Based on these trends we estimate existential risk
and recommend countermeasures to minimize existential risk with minimal socioeconimic damage.
}

\section{Introduction}

With current AI technologies, harm done by AIs is limited to power that we directly put in their
hands.  As said in \cite{yam2018timeline}, ``For Narrow AIs, safety failures are at the same level
of importance as in general cybersecurity, but for AGI it is fundamentally different.'' 

By combining
our existing knowledge on bad things that have happened and our wild guesses about things that
could happen, we can really do some solid guessing toward what kind of problems we could see in the
near future. Far future, maybe not, and this really doesn't work for AGI because it'll probably get
completely, categorically different from current AI. Like currently we're like uh oh we put
computers in charge of stocks and they blew it up. The point being that WE put them in charge of
stocks, and they stayed in the box. Once the AI can leave the box we assign it to (even for in-box
related goals) then these kind of failures become the only-make-them-once sort of failures. 

\section{Data Collection}

We collected data from \cite{yam2018timeline} and \cite{yam2019classification}, as well as from
online sources listed in the appendix and classfied it using a modified form of the schema presented
in \cite{yam2019classification}. 

DATA DATA DATDA DATA

MAYBE SOME CHARTS

\section{Data Analysis}

\marginpar{The data analysis is made up - still have to do this and will change is results differ}

Based on this data, we can model the severity of ai failures as a (some probability distribution)
and predict that a 1 million dollar failure has an expected wait time of 10 years, and a world
ending event it expected every 200 years at current levels, with a 10\% chance of happening in the
next 50 years.

Using the above model to try to tweak our situation to make these numbers better is a bad idea - we
would just overfit. The only solution is to stop all value producing activity.






\section{existing work}

\section{paper notes}

\cite{uesato2018adversarial}

20 pages, focused on rl training safe driverless card. Problem is that evaluating rl agaents
can take longer than training them, so add an adversary that evaluates them on the hard stuff,
like a driving instructor skipping straight to tight parallel parking when they see you're good

Doesn't actually list any failures

\cite{wallace2018landscapes}

33 pages. I like this already, taking the analogy of organizations as agents seriously.

This is a really complex paper, beyond me in a lot of ways but really neat.

\cite{anderson2005control}

20 pages. This is basically one big failure example, about a specific technique called ``adaptave
control". It also includes details about an overreaction to these failures, followed by ``don't use
it unless it's applicable" advice which has to be hammered into the naive buyers of COTS AI
software.

\cite{cook1998complex}

5 pages. \say{Post-accident attribution accident to a \say{root cause} is fundamentally wrong.}

Outch, okay yeah I get it. Damn.

\cite{carvin2017normal}

Uses \say{Charles Perrow’s Normal Accident Theory} which might be of use to me since I'm doing
similar analysis but for all of society instead of just autonamous weapons.

\cite{maas2018regulating}
\cite{humbatova2019taxonomy}

TODO get yam's papers in here

\subsection{categorizations and their data}

\section{unified categorization and how much data we got yeeeah}

So I like the tag based system, and title + link + short description is a pretty good way to go. The
tags all seem pretty distinct and nessesary, but are tricky to convert between.

\bibliography{mybib}{}
\bibliographystyle{plain}
\end{document}
